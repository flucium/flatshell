\documentclass{article}
\usepackage[utf8]{inputenc}
\usepackage[letterpaper, top=2cm, bottom=2cm, left=3cm, right=3cm, marginparwidth=1.75cm]{geometry}
\usepackage{titlesec}
\usepackage{color}
\title{FlatShellの開発}
\author{FURUKAWA \textless flucium@flucium.net\textgreater }
\date{April 23, 2024}
\begin{document}
\maketitle

\section*{はじめに}
Shellとは,オペレーティングシステム(OS)とユーザー間のインフェーフェイスとして機能するプログラムのことである.\newline
よく知られているものとして,Unix shellであるBashやZsh,MicrosoftのPowerShellなどがある.\newline
FlatShellとは,独自のShellであり,Unix shellなどには該当しない.また,Unix shellを比較対象としている.

\section{FlatShellの思想}
“フラットな状態は良い”という思想のもとで開発している.\newline
構文はもちろんのこと,処理系などの実装においても,可能な限りネストを深くしないように心掛けている.また,Unix shellの構文と比較して,自由な形式で書けることを意識している.

\section{字句解析器}
FlatShellでは,入力に対してプリプロセッシングという事前処理を多段階で行い,その結果に対してスキャナとトークナイザを走らせ,トークンを得るようにしている.\newline
以降,順を追って説明する.

\subsection{プリプロセッシング}
\begin{enumerate}
    \item 入力を受ける.
    \item 先頭から順にリードし,Sharp(\#)がリードされた時点で,LF,CR(\textbackslash n , \textbackslash r),Semicolon(;)まで読み飛ばす.
    \item 各行をみていく.空の行を読み飛ばしていく.(又は,空では無い行のみを拾い上げていく).
    \item LF,CRをSemicolonに置換する.
    \item 最後に,ここまで処理した文字列をcharのベクタに変換する.
\end{enumerate}
\subsection{スキャナ}
条件一致するまで,charのベクタをリードし続ける.リードされたcharは,順にバッファへと入れていく.条件一致した時点で処理を終了させる.\newline
バッファを1つ目の戻り値とする.条件一致したcharを2つ目の戻り値とする.条件一致したcharにポジションは移動していない.

\subsection{トークナイザ(小)}
対象ごとに,小さなトークナイザを定義する.
    \begin{enumerate}
        \item 文字列と仮定する.スキャナはWhitespace又はSYMBOLTABLEに当たらない限り,スキャンし続ける.スキャンし,得られたcharのベクタを文字列に変換し,String Tokenとする.\newline
        Whitespace及びSYMBOLTABLEに関する例外として,Double quote(")又はSingle quote(')の範囲は,全て文字列として解釈する.
        \item 数列と仮定する.Current charを確認し,10進数ならば数列と再仮定する.スキャナはWhitespace又はSYMBOLTABLEに当たらない限り,スキャンし続ける.スキャンし,得られたcharのベクタを文字列に変換する.文字列を数列にパースする.パースに成功したら,Number Tokenとする.
        \item 変数名と仮定する.Current charを確認し,Dollar(\$)ならば以降を変数名であると再仮定する.Dollar以降を対象とする.
        スキャナはWhitespace又はSYMBOLTABLEに当たらない限り,スキャンし続ける.スキャンし,得られたcharのベクタを文字列に変換する.文字列がアルファベットから始まる場合には,それをIdent Tokenとする.
        \item ファイルディスクリプタと仮定する.Current charを確認し,Att(@)ならば以降をファイルディスクリプタであると再仮定する.以降を対象とする.スキャナはWhitespace又はSYMBOLTABLEに当たらない限り,スキャンし続ける.スキャンし,得られたcharのベクタを文字列に変換する.文字列を数列にパースする.パースに成功したら,FD Tokenとする.
        \item 特殊文字の扱い.以下の通りである.
            \begin{enumerate}
                \item Semicolonならば,Semicolon Tokenとする.
                \item Equal(=)がならば,Equal Tokenとする.
                \item Ampersand(\&)ならば,Ampersand Tokenとする.
                \item Vertical bar(\textbar)ならば,Pipe Tokenとする.
                \item Greater than (\textgreater)ならば,Gt Tokenとする.(Version 0.0.1ではRedirectのオペレータとして使用する.)
                \item Less than(\textless)ならば,Lt Tokenとする.(Version 0.0.1ではRedirectのオペレータとして使用する.)
                \item Attならば,4.を試みる.4.に失敗した場合には,2つの条件で処理を変える.\newline
                条件1,ファイルディスクリプタをリードする関数そのものがエラーを返さずに,空の値(None)を返した場合には,End Of Fileとして解釈する.つまり,EOF Tokenとする.\newline
                条件2,ファイルディスクリプタをリードする関数がエラーを返した場合には,仮リード(peek char)を行い,その結果に基づいて処理を変える.仮リードの結果がSomeであり尚且つWhitespace,又は仮リードの結果が存在しない(None)場合には,Attを文字列として解釈する.つまり,String Tokenとする.\newline
                それ以外は,エラーとする.
                \item Dollarならば,3.を試みる.3.に失敗した場合には,2つの条件で処理を変える.\newline
                条件1,変数名をリードする関数そのものがエラーを返さずに,空の値(None)を返した場合には,End Of Fileとして解釈する.つまり,EOF Tokenとする.
                条件2,変数名をリードする関数がエラーを返した場合には,仮リードを行い,その結果に基づいて処理を変える.仮リードの結果がSomeであり尚且つWhitespace,又は仮リードの結果が存在しない(None)場合には,Dollarを文字列として解釈する.つまり,String Token.\newline
                それ以外は,エラーとする.
                \item Double quote又はSingle quoteならば,1.に準拠する.
            \end{enumerate}
    \end{enumerate}
\subsection{トークナイザ}
トークナイザは,スキャナやトークナイザ(小)よりも先にCurrent charを確認し,適切なトークナイザ(小)を呼び出す.\newline
もし,Current charがWhitespaceならば,Skipをする.

\section{構文解析器}
入力を受け取り,字句解析器に渡す.字句解析器は,スキャナとトークナイザを組み合わせて,入力をトークナイズする.\newline
構文解析器は,トークナイズによって得られたトークン列に対して解析を行う.

\subsection{ライトパーサ}
ライトパーサといわれる,対象を限定した小さなパーサ群を定義する.

\begin{enumerate}
    \item 文字列と仮定する.入力されたトークンがString Tokenならば,String Exprとする.
    \item 変数名と仮定する.入力されたトークンがIdent Tokenならば,Ident Exprとする.
    \item 数列と仮定する.入力されたトークンがNumber Tokenならば,Number Exprとする.
    \item ファイルディスクリプタと仮定する.入力されたトークンがFD Tokenならば,FD Exprとする.
    \begin{enumerate}
        \item Exprと仮定する.入力されたトークンがString,Ident,Number,FD Tokenのいずれかであれば,1.〜4.のライトパーサーにトークンを渡して,パースする.\newline
        また,このライトパーサをExpr パーサとする.対象外としたいトークンを指定することができる.つまり,条件一致でExprのパースを行うことができる.
        \item Assignと仮定する.入力として受け取れるトークンの数は3つであり,配列として受け取る.つまり,[Token;3]である.3は定数とする.\newline
        LeftはIdent Tokenであることが期待され,MiddleはEqual Tokenであることが期待される.RightはString,Number,FD Tokenであることが期待される.その上で,LeftとRightを適切なライトパーサに渡し,パースを行う.また,MiddleがEqualであるかも確認する.
        \item Redirectと仮定する.2つの形式を想定する必要がある.入力として受け取れるトークンの数は,2または3である.入力されたトークンの数を確認し,2なら形式ⅰ.へ,3なら形式ⅱ.へ渡す.
        \begin{enumerate}
            \item オペレータがLtならばLeftをFD0とし,GtならばLeftをFD1する.RightはExpr パーサへ入力し,Exprを得る.オペレータがLtまたはGtでなければ,エラーとする.
            \item オペレータがLtまたはGtであるかを確認し,そうでなければエラーとする.LeftはFD Tokenであることが期待され,RightはString, Number, Ident, FD Tokenであることが期待される.LeftとRightを適切なライトパーサへ渡して,パースする.
        \end{enumerate}
    \end{enumerate}
    \item コマンド列と仮定する.Command,Args,Redirects,Backgroundをそれぞれパースしていく.
    \begin{enumerate}
        \item Backgroundの判定.コマンド列と仮定されたトークン列の最後尾にAmpersandが存在するかを確かめ,存在する場合には,Ampersandをトークン列から除外し,尚且つCommandにBackgroundで処理するよう設定を行う.
        \item Commandのパース.String, Number, Ident Tokenであることが期待される.ライトパーサへ渡し,パースしExprを得る.
        \item Argsのパース.ArgsはOptionalである.Argsが0以上の場合には,ArgがString,Number, Ident Tokenであることが期待される.それぞれを適切なライトパーサへと渡し,パースする.
        \item Redirectsのパース.コマンドには複数のRedirectを含めることができる.Redirectsが0以上の場合には,Redirectを適切なライトパーサへ渡して,パースする.
    \end{enumerate}
    \item Pipeと仮定する.トークン列を受け取り,トークン列が0または1,2以上かで処理を変える.\newline
    トークン列にPipe Tokenが存在し,尚且つPipe Token以外の箇所をコマンド列と仮定するならば,このトークン列はPipe TokenをPivotとした,2次元のコマンド列であると解釈する.
    \begin{enumerate}
        \item トークン列が0.エラーとする.
        \item トークン列が1.尚且つPipe Tokenならエラーとする.
        \item トークン列が1.尚且つPipe Token以外ならバッファに入れ,バッファを戻り値とする.
        \item トークン列が2以上.Pipe TokenをPivotに,トークン列を再帰的に分割する.分割したトークン列をコマンド列であると仮定し,適切なライトパーサへと渡して,パースする.
    \end{enumerate}
\end{enumerate}
\subsection{パーサ}
入力を受け取り,字句解析器へ渡す.トークナイズを行い,トークン列を得る.得たトークン列の最後尾がEOF Tokenかを確認する.EOF Tokenならば,EOF Tokenを除外する.そうでなければ,エラーとする.\newline
Semicolon TokenをPivotに,トークン列を再帰的に分割する.このトークン列は,SemicolonをPivotとした2次元のトークン列であると解釈する.\newline
entries = [ [...], ...] となっている.\newline
各トークン列を,適切なライトパーサでパースしていく.パースに成功すると,Pipe,Assign,Commandのいずれかを得ることができる.それらをバッファに入れていく.バッファは,Semicolonを意味するASTのNodeである.そのバッファをルートとする.\newline
パースに成功した結果,ASTを返す.

\section{内部状態}
FlatShellでは,fsh-engineというライブラリ(crate)で処理系が定義され実装されている.\newline
また,その中でStateという構造体があり,Stateが内部状態を司ることになる.Stateは,Process handler,Current directory,Pipeのみを内包している.Stateとは別に,ShVarsというShell変数が定義され,実装されている.\newline
fsh-engineにはevalという関数があり,eval関数は,AST,State,ShVarsを入力として受け取る.ASTはeval関数の中で処理され,適切な処理系に渡される.Stateは状態を保持する.eval関数が終了しても,Stateの状態は保たれている.そのため,eval関数が予期しないエラーを引き起こしたとしても,eval関数とは別の場所でStateにイニシャライズ等をかけることができる.ShVarsに関しても同様である.

\section{Unix shellとの違い}

\subsection{Whitespaceの考え方}
FlatShellでは,可能な限りWhitespaceを認めることとしている.また,それが良いと考えている.\newline
なぜ,Whitespaceを認めた自由形式である方が良いと考えるのか? 最も,Bugを減らすことができるからである.Whitespaceを認めていない場合において,\$A=Helloとしたつもりが,\$A = HelloとTypoされていたとき,多くのUnix shellではエラーとして扱われる.\newline
しかし,処理系などの実装ミスによっては,\$A= Helloをコマンド等と解釈されてしまう可能性も否定できない.\newline
構文を緩くすることで,処理系などの実装を容易にし,実装ミスを減らすことが可能であると考えている.Whitespaceは,その1つである.

\subsection{Unix shellにおけるWhitespace}
なぜ,Unix shellではWhitespaceを認めていない箇所が多いのだろうか? または,変則的ともいえるような構文となっているのだろうか?\newline
主に2つの理由から,Whitespaceを認めていない思われる.
\begin{enumerate}
    \item WhitespaceをSkipする動作を必要とする.その動作が入る可能性があり,Whitespaceを認めていない場合とでは,少なからずリソース消費が多くなる. 
    \item 字句及び構文解析を容易にするため.例えば,A=Helloという入力を受け取ったとき,その時点ではAが何を意味する文字(または文字列)なのかを理解していない.Aの次をリードし,それがEqualならば,Aを変数名だと直ぐに確定させることができる.
\end{enumerate}

\subsection{Unix shellにおける変数の代入式}
代入式は,\textless Variable name\textgreater\textless Equal\textgreater\textless Value\textgreater となっている.\newline
具体的には, \colorbox{yellow}{A=Hello}と書く.\newline
ここには,Whitespaceを含めてはいけないという決まりがある.

\subsection{Unix shellにおける変数の参照}
参照は,\textless Dollar\textgreater\textless Variable name\textgreater となっている.\newline
具体的には,\colorbox{yellow}{\$A}や\colorbox{yellow}{\$PATH}と書く.\newline
勿論,Dollarと変数名の間にWhitespaceを入れることは認められない.これには同意である.\newline
Whitespaceを認めてしまうと,\textless Dollar\textgreater\textless Whitespace\textgreater\textless Variable name\textgreater となれば,Dollarがコマンド名なのか,変数名を意味するDollarなのか,または変数名なのかを判定するのが難しくなる.困難とまでは言わないが,無理に容認すれば思わぬBugにつながる可能性がある.場合によっては,セキュリティ上のリスクとなるかもしれない.

\section{コンピュータリソースの今と昔}
5.1,5.2で触れた通り,Unix shellではWhitespaceを認めていない箇所が多々ある.\newline
C言語の標準 APIが定義されたIEEE Std 1003.1-1988,Shellの仕様が追加されたPOSIX.2(つまり1992年)と2024年の今では,コンピュータリソースに大きな差がある.\newline
現代のコンピュータリソースを持ってして,WhitespaceのSkipに費やされるリソースを惜む必要はないだろう.特に,オプティマイズすることを前提とするならば,尚更である.

\section{今後について}
追加予定:分岐構造,反復構造,Shell変数とは独立した変数(Shell scriptで使用するため).

\bibliographystyle{plain}
\bibliography{}
\begin{enumerate}
    \item https://github.com/flucium/flatshell
\end{enumerate}
\end{document}
